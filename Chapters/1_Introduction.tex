% Chapter 1

\chapter{Introduction} % Main chapter title

\label{1_Introduction} % For referencing the chapter elsewhere, use \ref{Chapter1} 

\lhead{\emph{Introduction}} % This is for the header on each page - perhaps a shortened title

%----------------------------------------------------------------------------------------
\begin{comment}
The Introduction to the dissertation should set out the background to the research study and address the following areas:
The context in which the research took place
- What is the background, the context, in which the research took place?
- Why is this subject or issue important
- Who are the key participants and/or ‘actors’ in the area under investigation?
- Are there important trends or pivotal variables of which the reader needs to be made aware?

- A clear and succinct statement of the aims and objectives that the dissertation is going to address.
- Have you presented a clear and unambiguous exposition of your research aim, the objectives you will address to meet this aim and your research questions?

The reasons why this study was carried out
- Was this study undertaken for example in order to test some aspect of professional or business practice or theory or framework of analysis?
- Was the research carried out to fulfil the demands of a business organisation?


This chapter may be between 500 to 750 words although in some subjects or topics the justification of the subject and scope may change the length of this chapter.


Introduction – brief overview explaining the background and importance of the study

3. Statement of Problem – specifically what the researcher wants to know; 

4. Purpose of the Study – explanation of the problem and what the researcher hopes to achieve by conducting the study

5. Theoretical framework, research questions, or objectives – used to guide the direction of the research;
\end{comment}
%Supernovae

%CONTEXT - Astronomy
Future astronomical surveys will generate far too much data for astronomers to tackle as they have traditionally done.
It was the norm for astronomers to eyeball images of the sky, looking for interesting objects or transients.
\textit{Transients} are objects that appear for a short amount of time.
These include, but are not limited to, asteroids, gamma ray bursts, supernovae or objects unknown to science altogether.
Telescopic cameras used in surveys have limited exposure to areas within the night sky,due to its vastness.
When transients are discovered, follow ups with other telescopes necessary in order to accurately identify the object.
Not all transients are created equal; some, like supernovae, are more valuable to science than others.
Supernovae are rare gems for cosmologists which allow for learning about the history, curvature and content of the Universe.

Telescope technology has improved so much transients are discovered at an overwhelming rate.
Astronomers lack the resources to follow up on every one.

The amount of data will be so large that the mere storage of it becomes a technological challenge, never mind computational use of it.
In addition, there is increased interest in studies that use data from telescopes sensitive to different wavelengths.
Such \textit{multi-wavelength} astronomy exacerbates the problem of handling and computing large amounts of data.
To appreciate the extent of the problem, consider the upcoming Large Synoptic Survey Telescope (LSST) which will become scientifically active in 2022\citep{abell2009lsst}.
Over its ten year lifespan it is estimated to discover over 10 billion galaxies.
Each day around 30 TB of raw image data will be generated that will require processing and reduction\citep{kantor2005lsst}.

The final archive will be around 60 PB, with the final object catalogue approximately 10-20 PB\citep{borne2007machine}.
This challenge calls for a new approach to astronomy, one that is capable of dealing with Big Data.
\textit{Big Data} is a broad term for data sets so large that traditional data processing methods are inadequate.
This is a problem that is now facing many fields and industries.
Maintaining the status quo will be impossible; other tools must be explored.
%pivotal variable end opf moors law, serial becoming parallel


%CONTEXT - Machine Learning
As data challenges have been building, the computer science world has had a renewed interest in Machine Learning.
\textit{Machine Learning} (ML) is the capacity of a computer program to `learn' from data.
For example, many pictures of a person can be shown to an algorithm such that it learns how to identify him/her in another photo without the explicit instruction of a programmer.
The reinvigorated interest in Machine Learning has largely been the result of an exponential growth in computing power.
Newly discovered training methods are impractical to implement on older computers.
Recently Machine Learning has shown great success in artificial intelligence and Big Data applications,  from the popularly used cellphone assistant Siri\citep{Siri} to driver-less Google cars\citep{waldrop2015no}.
Convolutional Neural Networks (CNNs), a machine learning method loosely inspired by the brain, have achieved near-human level performance at image recognition in as little time as a third of a second\citep{taigman2014deepface}.
This task, easy for us, has evaded the capability of computers and computer programmers for decades.
CNNs are an example of \textit{Deep Learning} (DL) algorithms.
They are `deep' in the sense that they use a many-layered architecture for learning.
Deep Learning has had incredible success in other domains too, from voice recognition, brain-machine interfaces and medical diagnoses illustrating its promise for many applications.


%PROBLEM STATEMENT
%AIM and Research objectives
While Machine Learning has had plenty of application in many arenas, it is still fairly new to astronomy.
The aim of this study is to establish whether or not Machine Learning has a positive role to play for the future of astronomy.
To answer this problem there are several research objectives to lead us to our aim.
First we must come to understand Machine Learning fundamentals.
In particular, we must learn what Deep Learning is and how to successfully apply it.
Second, we will need to identify a problem we can use as test-case of DL's effectiveness in astronomy.
For this we will apply DL to transient classification of the Sloan Digital Sky Survey (SDSS) from a few years ago, to observe the accuracy and efficiency achieved on a real-world astronomy dataset.
As CNNs have proven successful at image-recognition, they are the ML method of choice for this application.
This will act as a useful playground for exploring the role machine learning has for the future of astronomy.
The final research objective is the evaluation of ML's future in astronomy.

%Research Questions
The primary interest is to ascertain the extent to which ML is effective. 
To be a viable solution, machine learning must be both accurate and efficient.
Classification accuracy needs to be competitive with trained astronomers.

As supernovae are quite rare, about once in a hundred years per galaxy, ML must also be very stringent as to what constitutes a supernova.
If not, then the pool of objects classified as supernovae will be contaminated with countless false-positives and imaging artefacts.
This will waste resources on unnecessary and expensive follow-ups.
The ML algorithm must be able to operate in real-time with a large data influx in order to be efficient.
DL promises to outperform more classical ML approaches\citep{du2014machine}.
We will investigate the validity of this claim.
From our application we will see what DL can tell us about the SDSS transient data.

%Thesis Structure
This thesis has been structured in the following way:
Chapter 1 will use Neural Nets as a drawing board for understanding the inner workings of machine learning methods and their successful application.
We will need to know the subtleties of what is involved in machine learning and know how to evaluate our results in an objective way.
Chapter 2 will layout the nature of the test case.
We will detail what the SDSS sky survey entails; what data is collected and how it is structured. 
Next, the current status of ML transient classification will be investigated.
It is useful to see what others have discovered in their research.
Chapter 3 lays out the research methodology.
This will tackle the question of how to best achieve the research objectives in a logically sound way.
Chapter 4 will provide an analysis and discussion of the results.
Finally we conclude with closing remarks and future recommendations.


