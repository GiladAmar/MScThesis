In the introduction to the dissertation you described the context of the
research. In the literature survey you analysed the work of previously
published authors and derived a set of questions that needed to be answered to
fulfil the objectives of this study. In the research methodology section you
showed the reader what techniques were available, what their advantages and
disadvantages were, and what guided you to make the choice you did. In the
results section, you present to the reader the outcome of the research exercise.
The introduction of this chapter reminds the reader what, exactly, were the
research objectives. Your review of the literature and your evaluation of the
various themes, issues and frameworks helped you to develop a more specific
set of research questions. In essence, your analysis of the data that you have
collected from your fieldwork should provide answers to these questions. You
should, as a matter of priority, focus attention on data that is directly relevant to
the research questions. You should avoid the mistake of including analysis that
might be interesting in a general way, but is not linked to the original direction
of the dissertation. Peripheral data can be included as an appendix, however
you are reminded that there is a limit of twenty-five pages for appendices.
The introduction should also explain how the results are to be presented.
13
This is the heart of the dissertation and must be more than descriptive.
This chapter develops analytic and critical thinking on primary results and
analysis with reference to theoretical arguments grounded in the literature
review. You should try to highlight where there are major differences and
similarities from the literature or between different groups. Where a model or
framework of analysis has been used or is being developed you should
highlight the main relationships as well as explaining the reason and
significance behind features or decisions being discussed.