% Chapter 1

\chapter{Research Design} % Main chapter title

\label{Chapter2} % For referencing the chapter elsewhere, use \ref{Chapter1} 

\lhead{Chapter 2. \emph{Research Design}} % This is for the header on each page - perhaps a shortened title

%----------------------------------------------------------------------------------------



\begin{comment}
You should begin the Research Methodology chapter by stating, again, the
research objectives of the project. This will enable the reader to make an
assessment as to the validity of your chosen research methodology. 
 
The methodology section should discuss what methods you are going to use in order to
address the research objectives of your dissertation.
Research Design – describes the methods that will be used to collect data or organize
creative products. May include the following depending on the department:
a. Description of the design
leave no question as to the procedures used to complete the study. 
You need to justify why the
chosen methods were selected as the most appropriate for your research, amongst the
many alternative ones, given its specific objectives, and constraints you may face in
terms of access, time and so on. Reference to general advantages and disadvantages of
various methods and techniques without specifying their relevance to your choice
decision is unacceptable. Remember to relate the methods back to the needs of your
research question.

b. Criteria for judging credibility and trustworthiness of results (where relevant)
9. Sampling – describe the aspects of the cases on which data collection and analysis will
focus (where relevant)
a. Indicate how access to the study population will be achieved
10. Variables – describe aspects of the cases on which data collection and analysis will focus
(where relevant)
11. Methods of Data Collection – explain how each variable will be measured (where
relevant)
12. Data Analysis Procedures – describe



The conclusion of this chapter should provide a summary of the main points
that have been covered. The conclusion should also direct the reader as to how
the contents of this chapter link in with the contents of the next chapter, your
findings. This chapter will be usually be between 1,000 and 2,000 words.
\end{comment}





%Data summary 
%Fits Files
%gri bands
%what is measured.



Many things to investigate

Diff/ and Search Images
Data normalization
No. Layers
filters per layer,
filter size
Pooling 
Dropout

Weight initializaton
learning rate
loss fucntion

Batch norm
optimizers
l1 and l2 regularization
patience
no epoch
activations
number of neurons in final layers
Batch size
Data Augmentation:
	


Binary vs Multiple Classifiaction


Computing resources
floating point operations
Theano
Keras
Sklearn
cnmem
fastmath
cudnn
skimage
theano
numpy
pyfits
tsne
buffering on CPU

SKA server (4 Titan X GPUs)


